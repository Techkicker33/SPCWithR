\begin{frame}

In statistical quality control, the individual/moving-range chart is a type of control chart 
used to monitor variables data from a business or industrial process for which it is impractical 
to use rational subgroups.[1]

\end{frame}
%======================================%
\begin{frame}
\frametitle{Control Charts}
The chart is necessary in the following situations:[2]:231

Where automation allows inspection of each unit, so rational subgrouping has less benefit.
Where production is slow so that waiting for enough samples to make a rational subgroup
unacceptably delays monitoring
\end{frame}
%======================================%
\begin{frame}
\frametitle{Control Charts}
For processes that produce homogeneous batches (e.g., chemical) where repeat measurements vary primarily 
because of measurement error
\end{frame}
%======================================%
\begin{frame}
\frametitle{Control Charts}
The "chart" actually consists of a pair of charts: one, the individuals chart, displays the individual measured values; 
the other, the moving range chart, displays the difference from one point to the next. 
As with other control charts, these two charts enable the user to monitor a process for shifts in the process that 
alter the mean or variance of the measured statistic.
\end{frame}
%======================================%
\end{document}