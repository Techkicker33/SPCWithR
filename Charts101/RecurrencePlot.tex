Recurrence plot
From Wikipedia, the free encyclopedia
In descriptive statistics and chaos theory, a recurrence plot (RP) is a plot showing, for a given moment in time, the times at which a phase space trajectory visits roughly the same area in the phase space. In other words, it is a graph of

\vec{x}(i)\approx \vec{x}(j),\,
showing i on a horizontal axis and j on a vertical axis, where \vec{x} is a phase space trajectory.

Contents  [hide] 
1	Background
2	Detailed description
3	Extensions
4	Example
5	See also
6	References
7	External links
Background[edit]
Natural processes can have a distinct recurrent behaviour, e.g. periodicities (as seasonal or Milankovich cycles), but also irregular cyclicities (as El Niño Southern Oscillation). Moreover, the recurrence of states, in the meaning that states are again arbitrarily close after some time of divergence, is a fundamental property of deterministic dynamical systems and is typical for nonlinear or chaotic systems (cf. Poincaré recurrence theorem). The recurrence of states in nature has been known for a long time and has also been discussed in early work (e.g. Henri Poincaré 1890).
