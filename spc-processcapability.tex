

\documentclass[11pt]{article} % use larger type; default would be 10pt

\usepackage[utf8]{inputenc} % set input encoding (not needed with XeLaTeX)

\usepackage{geometry} % to change the page dimensions
\geometry{a4paper} 
\usepackage{graphicx} % support the \includegraphics command and options

\usepackage{booktabs} % for much better looking tables
\usepackage{array} % for better arrays (eg matrices) in maths
\usepackage{paralist} % very flexible & customisable lists (eg. enumerate/itemize, etc.)
\usepackage{verbatim} % adds environment for commenting out blocks of text & for better verbatim
\usepackage{subfig} % make it possible to include more than one captioned figure/table in a 
\usepackage{fancyhdr} % This should be set AFTER setting up the page geometry
\pagestyle{fancy} % options: empty , plain , fancy
\renewcommand{\headrulewidth}{0pt} % customise the layout...
\lhead{Stats-Lab.com}\chead{Quality Management}\rhead{Summer 2014}
\lfoot{}\cfoot{\thepage}\rfoot{}

%%% SECTION TITLE APPEARANCE
\usepackage{sectsty}
\allsectionsfont{\sffamily\mdseries\upshape} 
\usepackage[nottoc,notlof,notlot]{tocbibind} % Put the bibliography in the ToC
\usepackage[titles,subfigure]{tocloft} % Alter the style of the Table of Contents
\renewcommand{\cftsecfont}{\rmfamily\mdseries\upshape}
\renewcommand{\cftsecpagefont}{\rmfamily\mdseries\upshape} % No bold!


\begin{document}

%---------------------------------------------------%
\section{Process Capability}

\subsection{What is a capable process?}

\subsection{Process Capability Indices}

Process Capability indices should be easy to compute and easy to understand. If the index is based on the assumption of normality, it must not be undermined by slight deviations from that assumption.

Process Capability Indices have evolved from simple but flawed indices to more complex indices which overcome some of the shortcomings of their predecessors.

\begin{enumerate}
\item $C_p$ and modified $C_p$  ( $C^\ast_p$ )
\item $C_{pm}$
\item $C_{pk}$
\item $C_{pmk}$
\end{enumerate}

%---------------------------------------------------%

The simplest capability index is $C_p$ which is defined as 
\[ C_p = \frac{USL - LSL}{6 \sigma}  \]

Modified $C_p$ (Hsiang and Taguchi, 1985 )
\[ C^{\ast}_p = \frac{USL - LSL}{6\tau}

\[ E(X-T)^2 = \sigma^2 + (\mu - T)^2 \]

%---------------------------------------------------%

\[ C_{pm} = \frac{C_p }{ \sqrt{1 + \left( \frac{\mu-T}{\sigma}\right)^2  }}

The most frequently used capability index is $C_{pk}$
\[ C_{pk} = \frac{Z_{min}}{3} \]

%---------------------------------------------------%

A modification of $C_pm$ is to replace $C_p$ with $C_{pk}$.
\[ C_{pmk} = \mbox{min} \left( \frac{\mu - LSL}{3\sqrt{ \sigma^2 + (\mu-T)^2 }   }  \;,\;   \frac{ USL - \mu}{3\sqrt{ \sigma^2 + (\mu-T)^2 } \right) \]

%---------------------------------------------------%

\end{document}
