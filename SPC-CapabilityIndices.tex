
\documentclass[12pt]{article}

\usepackage{framed}
\usepackage{amsmath}
\usepackage{amssymb}
\usepackage{graphics}
\usepackage{graphicx}
%opening
\title{Statistical Process Control}
%\author{MA4605}
%http://www.qualityamerica.com/Knowledgecenter/statisticalprocesscontrol/interpreting_process_capability.asp
\begin{document}
\section*{Process Capability}

To perform a capability analysis on a process, the process needs to be in statistical control. It is
usually assumed that the process characteristic being measured is normally distributed. 

This may be
checked out using tests for normality such as the Kolmogorov-Smirnov test or the
Anderson-Darling test. Process capability compares process performance with process requirements.

Process requirements determine specification limits. \textbf{LSL} and \textbf{USL} represent the lower specification limit and the upper specification limit.
The data used to determine whether a process is in statistical control may be used to do the
capability analysis. The \textit{\textbf{3-sigma}} distance on either side of the mean is called the process spread. 

The
mean and standard deviation for the proccss characteristic may be estimated from thc data gathered for
the statistical process control study.

\subsection*{Calculating Process Capability Indices}
\[ \hat{C}_P = \frac{\mbox{USL} - \mbox{LSL}}{6s}\]

\[ \hat{C}_{PK} = \mbox{min} \left[\frac{\mbox{USL} - \bar{x}}{3s},\frac{\bar{x} - \mbox{LSL}}{3s} \right] \]

\[ \hat{C}_{PM} = \frac{\mbox{USL} - \mbox{LSL}}{6\sqrt{s^2+(\bar{x}-T)^2}}\]
\bigskip

\subsection*{Process Potential Index ($C_P$ Index)}

The process potential index, or $C_P$, measures a process's potential capability, which is defined as the allowable spread over the actual spread. The allowable spread is the difference between the upper specification limit and the lower specification limit. The actual spread is determined from the process data collected and is calculated by multiplying six times the standard deviation, $s$. The standard deviation quantifies a process's variability. As the standard deviation increases in a process, the $C_P$ decreases in value. As the standard deviation decreases (i.e., as the process becomes less variable), the $C_P$ increases in value.


\subsection*{Interpreting the $C_P$ Index}
Assuming the process distribution is normal and the process average is exactly centered between the engineering requirements, a $C_P$ index of 1 would give a capable process. 

By convention, when a process has a $C_P$ value less than 1.0, it is considered potentially incapable of meeting specification requirements. Conversely, when a process $C_P$ is greater than or equal to 1.0, the process has the potential of being capable.

However, to allow a bit of room for \textit{\textbf{process drift}}, the generally accepted minimum value for $C_P$ is 1.33. In general, the larger $C_P$ is, the better. 

In a process qualified as a \textit{\textbf{Six Sigma process}} (i.e., one that allows plus or minus six standard deviations within the specifications limits), the $C_P$ is greater than or equal to 2.0.

\subsection*{Shortcomings of the $C_P$ Index}
The $C_P$ index has two major shortcomings. First, it cannot be used unless there are both upper and lower specifications. Second, it does not account for process centering. If the process average is not exactly centered relative to the engineering requirements, the $C_P$ index will give misleading results. In recent years, the $C_P$ index has largely been replaced by $C_{PK}$.

\subsection*{The Process Capability Index ($C_{PK}$ Index)}
The process capability index, or $C_{PK}$, measures a process's ability to create product within specification limits. $C_{PK}$ represents the difference between the actual process average and the closest specification limit over the standard deviation, times three.

\subsection*{The $C_{PM}$ Index}

A $C_{PM}$ of at least 1 is required, and 1.33 is preferred. $C_{PM}$ is closely related to $C_{P}$. The difference represents the potential gain to be obtained by moving the process mean closer to the target. Unlike $C_{PK}$, the target need not be the center of the specification range.

\end{document}