The Mahalanobis distance is a measure of the distance between a point P and a distribution D, introduced by P. C. Mahalanobis in 1936. It is a multi-dimensional generalization of the idea of measuring 
how many standard deviations away P is from the mean of D.



Defining the Mahalanobis distance
You can use the probability contours to define the Mahalanobis distance. The Mahalanobis distance has the following properties:

It accounts for the fact that the variances in each direction are different.
It accounts for the covariance between variables.
It reduces to the familiar Euclidean distance for uncorrelated variables with unit variance.
