%- http://support.minitab.com/en-us/minitab/17/topic-library/quality-tools/acceptance-sampling/acceptance-sampling-graphs/operating-characteristic-oc-curve/


%==============================%
\begin{frame}
\frametitle{Acceptance Sampling}
\large

\begin{itemize}
\item The acceptable quality limit (AQL) is the worst tolerable process average (mean) in percentage or ratio that is still considered acceptable; that is, it is at an acceptable quality level. 
\end{itemize}

\end{frame}

%==============================%
\begin{frame}
\frametitle{Acceptance Sampling}
\large
\begin{itemize}
\item 
‘AQL’ stands for ‘Acceptance Quality Limit’, and is defined as the “quality level that is the worst tolerable” in ISO 2859-1. 
\item It decides on the maximum number of defective units, beyond which a batch is rejected. Companies usually set different AQLs for critical, major, and minor defects.
\item
For example: “I want no more than 1.5% defective items in the whole order quantity, on average over several production runs with that supplier” means the AQL is 1.5%.
\end{itemize}
\end{frame}

%==============================%
\begin{frame}
\frametitle{Acceptance Sampling}
\large
In practice, three types of defects are distinguished. For most consumer goods, the limits are:

\begin{itemize}
\item
0\% for critical defects (totally unacceptable: a user might get harmed, or regulations are not respected).
\item 2.5\% for major defects (these products would usually not be considered acceptable by the end user).
\item
4.0\% for minor defects (there is some departure from specifications, but most users would not mind it).
\end{itemize}

\end{frame}

%==============================%
\begin{frame}




\frametitle{Attribute acceptance sampling plans}

\begin{itemize}
\item Attribute sampling plans are based on the number of defects or the number of defective items in a sample. 
\item For example, how many defects are found in a sample of carpet, or how many bruised apples are in my sample.
\end{itemize}

\end{frame}
