 

MA4707 Syllabus
Exam Questions (Autumn 2007/2008)
Exam Questions (Autumn 2008/2009)
TQM and Six Sigma
European Foundation for Quality Management
ISO 9000 and ISO 9001
The Taguchi Loss Function
 
MA4707 Syllabus 
 
History of Quality: Western & Japanese approaches & the Gurus 
Quality organisation: policy, objectives, systems 
Total Quality Management: customers, employee participation, PDCA, societal learning 
Quality Systems: ISO, EFQM, MBNQA 
Cost of Quality: prevential, appraisal, failure, Taguchi loss function & traditional cost concepts 
 
Exam Questions (Autumn 2007/2008)
 
MA4707 Quality Management
 
1. Juran insists that the customer is the final arbiter of what quality is or is not with respect to a particular product or service. Outline his strategic approach. (20)
 
2. Outline Seddon’s case against the ISO 9000 standard. (20)
 
3. Define Quality Function Deployment and explain how it is used. (20)
 
4. How can TQM be used to create competitive advantage in business? (20)
 
 
Exam Questions (Autumn 2008/2009)
 
MA4707 Quality Management
 
1.Quality costs are generated in a variety of ways. Outline the main cost categories and describe how the impact of each category may be minimised.(20)
 
2.Explain Garvin’s eight product quality dimensions using examples.(20)
 
3.Describe the key aspects of EFQM Award.(20)
 
4.Outline the benefits of Continuous Improvement and how it may be achieved.(20)
 
 
 
TQM and Six Sigma
The Six Sigma management strategy originated in 1986 from Motorola’s drive towards reducing defects by minimizing variation in processes. 
The main difference between TQM and Six Sigma (a newer concept) is the approach. 
At its core, Total Quality Management (TQM) is a management approach to long-term success through customer satisfaction.
 
In a TQM effort, all members of an organization participate in improving processes, products, services and the culture in which they work.
The methods for implementing this approach come from the teachings of such quality leaders as Philip B. Crosby, W. Edwards Deming, Armand V. Feigenbaum, Kaoru Ishikawa and Joseph M. Juran.
 
A core concept in implementing TQM is Deming’s 14 points, a set of management practices to help companies increase their quality and productivity:
Create constancy of purpose for improving products and services.
Adopt the new philosophy.
Cease dependence on inspection to achieve quality.
End the practice of awarding business on price alone; instead, minimize total cost by working with a single supplier.
Improve constantly and forever every process for planning, production and service.
Institute training on the job.
Adopt and institute leadership.
Drive out fear.
Break down barriers between staff areas.
Eliminate slogans, exhortations and targets for the workforce.
Eliminate numerical quotas for the workforce and numerical goals for management.
Remove barriers that rob people of pride of workmanship, and eliminate the annual rating or merit system.
Institute a vigorous program of education and self-improvement for everyone.
Put everybody in the company to work accomplishing the transformation.
 
The term total quality management has lost favor in the United States in recent years; quality management is commonly substituted. Total quality management, however, is still used extensively in Europe.
 
European Foundation for Quality Management
 
EFQM (formerly known as the European Foundation for Quality Management) is a non-profit membership foundation based in Brussels. EFQM is the custodian of the EFQM Excellence Model, a non-prescriptive management framework that is widely used in public & private sector organisations throughout Europe and beyond.
EFQM Membership is open to organisations, rather than individuals. Members include: BMW, EDF, Grundfos, Philips, Ricoh, Robert Bosch,Solvay and Trimo.
EFQM runs the annual EFQM Excellence Award, which is designed to recognised organisations that have achieved an outstanding level of sustainable excellence.
 
ISO 9000 and ISO 9001
The ISO 9000 family of standards relate to quality management systems and are designed to help organizations ensure they meet the needs of customers and other stakeholders (Poksinska et al, 2002 ).
 
The standards are published by ISO, the International Organization for Standardization and available through National standards bodies. 
ISO 9000 deals with the fundamentals of quality management systems (Tsim et al, 2002), including the eight management principles (Beattie and Sohal, 1999; Tsim et al, 2002) 
on which the family of standards is based. 
 
ISO 9001 deals with the requirements that organizations wishing to meet the standard have to fulfill.
Third party certification bodies provide independent confirmation that organizations meet the requirements of ISO 9001. 
Over a million organizations worldwide are independently certified, making ISO 9001 one of the most widely used management tools in the world today.
 
 
The Taguchi Loss Function
 
The Taguchi Loss Function is a graphical depiction of loss developed by the Japanese 
business statistician Genichi Taguchi to describe a phenomenon affecting the value of 
products produced by a company. 
 
Praised by Dr. W. Edwards Deming (the business guru of the 1980s American quality movement), it 
made clear the concept that quality does not suddenly plummet when, for instance, a machinist 
exceeds a rigid blueprint tolerance. Instead "loss" in value progressively increases as variation 
increases from the intended condition. This was considered a breakthrough in describing quality, and helped fuel 
the continuous improvement movement that since has become known as lean manufacturing.

What is a quality guru?
A guru, by definition, is a good person, a wise person and a teacher. A quality guru should be all of these,
plus have a concept and approach to quality within business that has made a major and lasting impact. The
gurus mentioned in this section have done, and continue to do, that, in some cases, even after their death.
 
The gurus
There have been three groups of gurus since the 1940’s:
Early 1950’s Americans who took the messages of quality to Japan
Late 1950’s Japanese who developed new concepts in response to the Americans
1970’s-1980’s Western gurus who followed the Japanese industrial success
 
A brief overview of their contribution to the quality journey is given, supported by several references.
 
The Americans who went to Japan:
W Edwards Deming placed great importance and responsibility on management, at both the individual and company level, believing management to be responsible for 94% of quality problems. His fourteen point plan is a complete philosophy of management, that can be applied to small or large organisations in the
public, private or service sectors.
 
He believed that adoption of, and action on, the fourteen points was a signal that management intended to stay in business. Deming also encouraged a systematic approach to problem solving and promoted the widely known Plan, Do, Check, Act (PDCA) cycle. The PDCA cycle is also known as the Deming cycle, although it was developed by a colleague of Deming, Dr Shewhart.
 
It is a universal improvement methodology, the idea being to constantly improve, and thereby reduce the  difference between the requirements of the customers and the performance of the process. The cycle is about learning and ongoing improvement, learning what works and what does not in a systematic way; and the cycle repeats; after one cycle is complete, another is started.
 
Dr Joseph M Juran developed the quality trilogy – quality planning, quality control and quality  improvement. Good quality management requires quality actions to be planned out, improved and controlled. The process achieves control at one level of quality performance, then plans are made to improve the performance on a project by project basis, using tools and techniques such as Pareto analysis. This activity eventually achieves breakthrough to an improved level, which is again controlled, to prevent any deterioration.
 
Breakthrough Pareto Analysis
Holding the gains
project-by-project
 
Plan what is needed
Do it
Check that it works
Act to correct any problems or improve performance
 
The Japanese:
Dr Kaoru Ishikawa made many contributions to quality, the most noteworthy being his total quality viewpoint, company wide quality control, his emphasis on the human side of quality, the Ishikawa diagram and the assembly and use of the “seven basic tools of quality”:
 
• Pareto analysis - which are the big problems?
• Cause and effect diagrams - what causes the problems?
• Stratification - how is the data made up?
• Check sheets - how often it occurs or is done?
• Histograms  - what do overall variations look like?
• Scatter charts  - what are the relationships between factors?
• Process control charts - which variations to control and how?
 
He believed these seven tools should be known widely, if not by everyone, in an organisation and used to analyse problems and develop improvements. Used together they form a powerful kit. Shigeo Shingo is strongly associated with Just-in-Time manufacturing, and was the inventor of the single minute exchange of die (SMED) system, in which set up times are reduced from hours to minutes, and the Poka-Yoke (mistake proofing) system. In Poka Yoke, defects are examined, the production system stopped and immediate feedback given so that the root causes of the problem may be identified and prevented from occurring again. The addition of a checklist recognises that humans can forget or make mistakes!
 
Links
 
http://www.businessballs.com/dtiresources/total_quality_management_TQM.pdf
 
 
