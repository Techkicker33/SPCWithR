
Acceptance Sampling

Acceptance sampling is a method used to determine if merchandise or other items meet certain standards. This kind of sampling is employed in different manufacturing venues and often favored because it can give manufacturers specific control over what type of standards to exert. Various methods are utilized to determine how to sample from batches, what percentage of error is acceptable, and when to discard batches based on finding too many errors.


One reason acceptance sampling is useful is because it avoids taking an entire batch of product, testing it, and then discarding it. Once tested, many items aren’t saleable anymore and must be discarded. By testing some of a batch, determined either with random samples, simple random samples or another form of selection guaranteeing random picks, most of the batch can still be used, if deemed acceptable.


The easiest way to illustrate acceptance sampling is through example. Company Z wants to test their candy hearts to make certain that no more than 10% of the product manufactured is broken. With a batch of candy hearts, they determine that they will randomly select 5% of that batch and see if any of the hearts are broken. If that smaller sample has more than 10% broken hearts, Company Z will discard the entire batch. With less than 10% broken hearts, the batch is considered good, is packaged and sent to stores.


There are a few things to note about this example. Company Z gets to decide percentage of defects it will accept. It could easily accept up to 20% broken hearts, or could set a more stringent standard and only accept 3%. The company also gets to determine how to select samples from the batch and the sample size. Many companies run such sampling on every batch they produce to achieve consistent quality control.


There are specific statistical operations that can help companies determine sample size needed that is representative of the batch. In other words, companies must select a size that will be big enough to truly represent their population or the batch from which they come. This number is small enough that there are still plenty of leftover saleable pieces in the batch.


Acceptance sampling can be a very good way to determine the quality of a batch, given consideration to sample size and after setting a reasonable quality percentage. There are circumstances under which companies will draw a random bad sample, which could mean discarding a batch or selling a product that is more flawed than expected. Most frequently, the sample is representative of the batch and will serve as a good way to test quality.


The degree to which this means a company delivers a quality product really is based on the company determining acceptable level of error. With higher percentages of error allowed, quality will be poorer. Using lower error thresholds means most batches produced are of greater quality. In any type of equipment that is extremely delicate, companies may set low percentages for batch error in acceptance sampling.
