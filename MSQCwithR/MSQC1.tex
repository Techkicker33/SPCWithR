	\documentclass[a4paper,12pt]{article}
%%%%%%%%%%%%%%%%%%%%%%%%%%%%%%%%%%%%%%%%%%%%%%%%%%%%%%%%%%%%%%%%%%%%%%%%%%%%%%%%%%%%%%%%%%%%%%%%%%%%%%%%%%%%%%%%%%%%%%%%%%%%%%%%%%%%%%%%%%%%%%%%%%%%%%%%%%%%%%%%%%%%%%%%%%%%%%%%%%%%%%%%%%%%%%%%%%%%%%%%%%%%%%%%%%%%%%%%%%%%%%%%%%%%%%%%%%%%%%%%%%%%%%%%%%%%
\usepackage{eurosym}
\usepackage{vmargin}
\usepackage{framed}
\usepackage{amsmath}
\usepackage{graphics}
\usepackage{epsfig}
\usepackage{subfigure}
\usepackage{enumerate}
\usepackage{fancyhdr}

\setcounter{MaxMatrixCols}{10}
%TCIDATA{OutputFilter=LATEX.DLL}
%TCIDATA{Version=5.00.0.2570}
%TCIDATA{<META NAME="SaveForMode"CONTENT="1">}
%TCIDATA{LastRevised=Wednesday, February 23, 201113:24:34}
%TCIDATA{<META NAME="GraphicsSave" CONTENT="32">}
%TCIDATA{Language=American English}

\pagestyle{fancy}
\setmarginsrb{20mm}{0mm}{20mm}{25mm}{12mm}{11mm}{0mm}{11mm}
\lhead{MS4222} \rhead{Kevin O'Brien} \chead{Normal Distribution} %\input{tcilatex}

\begin{document}
%%  - Chapter 2 / Page 17

\section{ Multivariate Control Charts }
With the enhancements in data acquisition systems it is usual to deal with processes
with more than one correlated quality characteristic to be monitored. A common
practice is to control the stability of the process using univariate control charts. This
practice increases the probability of false alarm of special cause of variation.
Therefore, the analysis should be performed through a multivariate approach;
that is, the variables must be analyzed together, not independently.
In this chapter we present the multivariate normal distribution, the data structure
of the multivariate problems dealt in this book, the mult.chart function that allows
the computation in R, and the most used multivariate control charts:

\begin{itemize}
\item The control ellipsoid or w2 control chart
\item The T2 or Hotelling chart
\item The Multivariate Exponentially Weighted Moving Average (MEWMA) chart
\item The Multivariate Cumulative Sum (MCUSUM) chart
\item The chart based on Principal Components Analysis (PCA)
\end{itemize}


%-----------------------------------------------%

\subsection{2.1 The Multivariate Normal Distribution}
The multivariate normal distribution (MVN) is the core of the multivariate statistical
analysis. This is due to the fact that the sampling distributions of multivariate
distributions exhibit approximately normality due to the central limit theorem.
In the univariate case if a random variable is normally distributed with mean m
and variance s2 it has a density function:

\[  EQUATION   \]
where $-1<x<1$:

% E. Santos-Ferna´ndez, Multivariate Statistical Quality Control Using R,
% SpringerBriefs in Statistics 14, DOI 10.1007/978-1-4614-5453-3_2,
% # Springer Science+Business Media New York 2012


%====================================%


%=====================================%

Example 2.1
In order to perform in R a graphical representation of a bivariate normal distribution
with mean vector m = 0
0
 
and covariance matrix S = 10 3
3 6
 
we have

\begin{verbatim}
> mu <- c(0,0)
> sigma <- matrix(c(10,3,3,6),2,2)
> rho <- sigma[1,2] / (sqrt(sigma[1,1] * sigma[2,2]))
> # Defining the mean vector, the covariance matrix, 
>  # and the correlation coefficient:
> var1 <- seq(-12,12,.7)
> var2 <- var1
> f <- matrix(0, length(var1), length(var1))
> for( i in 1:length(var1)){
> for(j in 1:length(var1)){
> f[i,j] <- 1/(2 * pi * sqrt(sigma[1,1] * sigma[2,2] * (1-rho ^     2)))*exp(-1 /
(2 * (1-rho ^ 2)) * ((var1[i] - mu[1]) ^ 2 / sigma[1,1] + (var2[j] - mu[2]) ^ 2 /
sigma[2,2]-2 * rho * ((var1[i] - mu[1]) * (var2[j] - mu[2])) / (sqrt(sigma[1,1]) *
sqrt(sigma[2,2]))))}}
\end{verbatim}

%% MSQCChap2RCodeOutput1
\begin{framed}
\begin{verbatim}
> persp(var1, var2, f, xlab = "Variable 1", ylab = "Variable 2", zlab = "f(var1,
var2)", theta = 30, phi = 30, r = 50, d = 0.2, expand = 0.6, ltheta = 90, lphi =
180, nticks = 4)

\end{verbatim}
\end{framed}
Then R shows the bivariate density function (Fig. 2.1a).
Moreover it is possible to represent in a two-dimensional form using a contour
plot (Fig. 2.1b):
> contour(var1, var2, f, xlab = "Variable 1", ylab = "Variable 2", nlevels = 8,
drawlabels = F, xlim = c(-8,8), ylim = c(-8,8))


%--------------------------------------------------------------%


%--------------------------------------------------------------%
\section{2.3 \texttt{The mult.chart} Function}
The performing of the multivariate control chart in R can be carried out with the
function \texttt{mult.chart} which is a general function that allows to compute the most
accepted and diversified continuous multivariate chart such as

\begin{itemize}
	\item w2
	\item Hotelling T2
	\item MEWMA
	\item MCUSUM according to Crosier (1988)
	\item MCUSUM by Pignatiello and Runger (1990)
\end{itemize}

%% 2.3 The mult.chart Function 

%% Page 21


%--------------------------------------------------------------%

“mewma,” “\texttt{mcusum},” or “\texttt{mcusum2}” in the same order previously introduced.
For more details about the function see the package manual:
\begin{verbatim}
> help(package = "MSQC")
\end{verbatim}

In the function x must be a matrix or an array and jointly with type are the only compulsory arguments.

Other important functionalities are the Phase that can be I or II (being I for default) and the significance level (alpha) fixed in 0.01.

As it is shown in the next section, the covariance matrix (S) and mean vector (Xmv) can be entered to be used in Phase II.

Finally the function \texttt{mult.chart} returns:
\begin{itemize}
\item  The $T^2$ statistics
\item The Upper Control Limit (UCL)
\item The sample covariance matrix (S)
\item The mean vector (Xmv)
\item And if any point falls outside of the UCL and its decomposition
\end{itemize}
The execution of the function takes few hundredth of a second as can be tested by
\begin{verbatim}
> system.time(mult.chart(dowel1, type = "chi", alpha = 0.05))
\end{verbatim}



\subsection*{Example 2.2}
To illustrate the construction of an ellipsoid contour consider the dataset called
dowel that comprises 40 samples from two correlated quality characteristics (diameter
and length) collected from a manufacturing process of a dowel pin.



To call the dataset, just use

> data("dowel1")

The construction of the control ellipse for dowel1 results as follows. Setting the significance level:
\begin{verbatim}
 alpha <- 0.05 and
 p <- ncol(dowel1)
 # Estimate mean vector and the covariance matrix
 Xmv <- colMeans(dowel1)
 
\end{verbatim}
The function colMeans was used directly due to the fact that this is a problem of
individual observations:

> S <- covariance(dowel1)
So we have
m0 = ½ 0:50 1:00  and S = 4.91e - 05 8:58e - 05
8.58e - 05 4.20e - 04
 
.
The computation of the eigenvalues and eigenvectors is based on the R function
\begin{verbatim}
eigen:
> DDe <- eigen(S)$values
> Ue <- eigen(S)$vectors
\end{verbatim}

For more details see help function.
Then we have

l0 =½4:39e-04 3:02e-05, e1
0 =½0:22 -0:98, and e2
0 =½-0:98 0:22.

Plotting the ellipsoid origin given by Xmv. (at 0.50, 1.00) with the respective
axes labels and ranges:
\begin{framed}
\begin{verbatim}
> plot(Xmv[1], Xmv[2], xlim = c(0.46,0.54), ylim = c(0.95,1.06), xlab = "diameter",ylab = "length",pch = 3)
\end{verbatim}
\end{framed}
The direction of the ellipsoid axes is given by the eigenvectors:
> inc <- atan ((Xmv[2] + Ue[2,1] - Xmv[2]) / (Xmv[1] + Ue[1,1] - Xmv[1]))
Then we must compute the lengths regarding the x- and y-axes as follows:
\begin{framed}
\begin{verbatim}
> b <- (sqrt(DDe[1]) * sqrt(qchisq(1 - alpha,p))) * sin(inc)
> a <- (sqrt(DDe[1]) * sqrt(qchisq(1 - alpha,p))) * cos(inc)
> d <- (sqrt(DDe[2]) * sqrt(qchisq(1 - alpha,p))) * sin(inc)
> c <- (sqrt(DDe[2]) * sqrt(qchisq(1 - alpha,p))) * cos(inc)
\end{verbatim}
\end{framed}
Finally, we trace the axes using
\begin{framed}
\begin{verbatim}
> arrows(Xmv[1], Xmv[2], Xmv[1] + a, Xmv[2] + b)
> arrows(Xmv[1], Xmv[2], Xmv[1] - a, Xmv[2] - b)
> arrows(Xmv[1], Xmv[2], Xmv[1] - d, Xmv[2] + c)
> arrows(Xmv[1], Xmv[2], Xmv[1] + d, Xmv[2] - c)
\end{verbatim}
\end{framed}
%================================================%

The ellipse results by connecting the axes extremes.
Fortunately it is relatively easy to draw an ellipse in R, making use of this
algorithm:
> angle <- seq(0, 2 * pi, length.out = 200)
> ch <- cbind(sqrt(qchisq(1 - alpha,2)) * cos(angle), sqrt(qchisq(1 - alpha,2)) *
sin(angle))
> lines(t(Xmv - ((Ue %*% diag(sqrt(DDe))) %*% t(ch))),type = "l")
Figure 2.3a shows the result.
This procedure is known as confidence ellipsoid. Figure 2.3b shows the addition
of the points of the dowel1 array:
> points(dowel1)
Obtaining no points outside the ellipse, there is no evidence of special causes;
therefore the process is in-control. Notice that if the limits from the univariate
individual control chart are plotted, how much this area differs to the confidence
ellipse. In fact, four points fall outside to this area (Fig. 2.4).
The difficulty to identify the points beyond the confidence ellipsoid is one of the
main drawbacks of the tool, although it can be solved by inserting the sample
number in plot when the amount of points is not large.
Another disadvantage is the complexity to construct the ellipsoid when p > 2
which can be solved using the w2 control chart that results by plotting the test
statistics:
\[EQUATION\]

;p; (2.19)
where n is the sample size and the upper control limit:
UCL = w2
a;p: (2.20)
0.47 0.48
a b
0.49 0.50 0.51 0.52 0.53
0.95 1.00 1.05
diameter
length
0.47 0.48 0.49 0.50 0.51 0.52 0.53
0.95 1.00 1.05
diameter
length
Fig. 2.3 (a) Confidence ellipse with the axes for the dowel1 dataset. (b) Scatterplot for the dowel1
dataset with the confidence ellipse
%% - 24 2 Multivariate Control Charts
%=================================================%

When m and S are estimated through a sufficiently large sample then the w2 chart
can be used although the parameters are unknown.
Through the function mult.chart
> mult.chart(dowel1, type = "chi", alpha = 0.05)
The function returns (Fig. 2.5):
Showing results alike to the control ellipsoid. An advantage of this chart is that it
allows the evolution of the samples along time.
0.47 0.48 0.49 0.50 0.51 0.52 0.53
0.94 0.96 0.98 1.00 1.02 1.04 1.06
diameter
length
LCL UCL
LCL
UCL
Fig. 2.4 Scatterplot for
the dowel1 dataset with
the confidence ellipse
and the Shewhart control
limits
0 10 20 30 40
0
1
2
3
4
5
6
Sample
T2
UCL= 5.99
[1] "Chi-squared Control Chart"
$ucl
[1] 6
$t2
[1,] 1.61
[2,] 0.30
…
[39,] 1.58
[40,] 1.64
$Xmv
[1] 0.50 1.00
$covariance
[,1] [,2]
[1,] 4.91e -05 8.59e-05
[2,] 8.59e -05 4.20e-04
Fig. 2.5 w2 control chart for the dowel1 dataset
2.4 Contour Plot and w2 Control Chart 25
%% kevin.obrien@ul.ie

%===============================================================%
%===============================================================%

%% Page 27

Finally they are passed in the function mult.chart:
> mult.chart(dowel2, type = "chi", Xmv = vec, S = mat, alpha = 0.05)
The fourth sample falls beyond the UCL; as a consequence, there is evidence of
special causes, and then the process is out-of-control (Fig. 2.7).
\section*{2.5 Hotelling T2 Control Chart (Phase I)}
The origin of the T2 control chart dates back to the pioneer works of Harold Hotelling
who applied this method to the bombsight problem in Second World War. The
Hotelling (1947) procedure has become without doubt the most applied in multivariate
process control and it is the multivariate analogous of the Shewhart control chart.
For that reason, it is also known as multivariate Shewhart control chart.
Often in practice the parameters m and S are unknown and consequently must be
estimated across the unbiased estimators -x and S. Based on the multivariate
generalization of the t statistic from univariate normal theory:
t = -x - m
S=
ffiffiffi
n
p (2.21)
making
t2 =
ð-x - mÞ2
S2=n
= nð-x - mÞ S2  -1ð-x - mÞ; (2.22)
0 5 10 15 20 25 30
0
2
4
6
8
Sample
T2
UCL= 5.99
Fig. 2.7 w2 control chart
in Phase II for the dowel2
dataset
2.5 Hotelling T2 Control Chart (Phase I) 27
kevin.obrien@ul.ie

so the generalization results in
T2 = nðX - XÞ0ðSÞ-1ðX - XÞ (2.23)
with - X and S being the vector of means and the covariance matrix, respectively.
The statistics T2 follows an F distribution with p and (mn - m - p + 1) degrees
of freedom. Therefore for establishing the control in Phase I the UCL results in
UCL = pðm - 1Þðn - 1Þ
mn - m - p þ 1
Fa; p; mn-m-pþ1: (2.24)
While for monitoring future observations (Phase II) the limit is given by
UCL = pðm þ 1Þðn - 1Þ
mn - m - p þ 1
Fa; p; mn-m-pþ1: (2.25)
Here, (2.25), the number of samples (m) refers to the preliminary samples taken
to establish the in-control state (Phase I). Notice that this chart lacks lower control
limits (LCL) analogously to the w2 chart.
This chart is employed in introductory multivariate studies and has a good
performance in detection of large shifts in the mean.
According to Lowry and Montgomery (1995) the application of this chart
requires a number of quality characteristics between 2 and 10, taking more than
20 samples (often more than 50) of size 2, 3, or 10. These values are sometimes
limited by the very nature of the problem, though.
The following example explains the construction of this chart.
Example 2.3
In the manufacturing process of a specific carbon fiber tubing three correlated
quality characteristics are measured: inner diameter, thickness, and length of the
tubes in inches. The dataset named carbon1 contains the information of 30 samples
of size 8 taken and summarized in Table 2.1.
The sample mean vector, sample covariance, and correlation matrix result as
follows:
\begin{verbatim}
-x =
0:99
1:04
49:98
2
4
3
5; S  100 =
0:25 0:36 0:67
0:36 1:45 1:02
0:67 1:02 5:92
2
4
3
5; r =
1 0:63 0:57
0:63 1 0:38
0:57 0:38 1
2
4
3
5:
\end{verbatim}
It can be easily appreciated the direct correlation among the variables; being
significant between the inner diameter with the others.
As we are in the presence of a trivariate process, it is possible a spatial
representation. Figure 2.8 shows the three-dimensional scatterplot with the 99%
ellipsoid. All the points of the swarm are inside the ellipsoid.
A scatterplot matrix is presented below and corroborates the information offered
by the correlation matrix about the direct correlation between variables (Fig. 2.9)):
pairs(carbon1,labels=c("inner diameter", "thickness", "length"))
28 2 Multivariate Control Charts
\end{document}
%=======================================================%
