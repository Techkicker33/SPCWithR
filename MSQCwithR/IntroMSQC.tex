%%-https://arxiv.org/ftp/arxiv/papers/0901/0901.2880.pdf

The practitioner can not really study more than two or three charts to maintain process or product quality. It is very
helpful that in practice, only a few events are driving a process at any one time; different combinations of these
measurements are simply reflections of the same underlying events. 


Multivariate SPC refers to a set of advanced techniques for the monitoring and control of the operating performance
of batch and continuous processes. More specifically, multivariate SPC techniques reduce the information
contained within all of the process variables down to two or three composite metrics through the application of

Hotelling's T squared

Hotelling's T2 distribution is a multivariate method that is the multivariate counterpart of Student's t and which also forms the basis for certain multivariate control charts is based on Hotelling's T2 distribution, which was introduced by Hotelling (1947).

