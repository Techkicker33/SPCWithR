

\documentclass[11pt]{article} % use larger type; default would be 10pt

\usepackage[utf8]{inputenc}

%%% PAGE DIMENSIONS
\usepackage{geometry} % to change the page dimensions
\geometry{a4paper} % or letterpaper (US) or a5paper or....
% \geometry{margin=2in} % for example, change the margins to 2 inches all round
% \geometry{landscape} % set up the page for landscape
%   read geometry.pdf for detailed page layout information

\usepackage{graphicx} % support the \includegraphics command and options

% \usepackage[parfill]{parskip} % Activate to begin paragraphs with an empty line rather than an indent

%%% PACKAGES
\usepackage{booktabs} % for much better looking tables
\usepackage{array} % for better arrays (eg matrices) in maths
\usepackage{paralist} % very flexible & customisable lists (eg. enumerate/itemize, etc.)
\usepackage{verbatim} % adds environment for commenting out blocks of text & for better verbatim
\usepackage{subfig} % make it possible to include more than one captioned figure/table in a single float
% These packages are all incorporated in the memoir class to one degree or another...

%%% HEADERS & FOOTERS
\usepackage{fancyhdr} % This should be set AFTER setting up the page geometry
\pagestyle{fancy} % options: empty , plain , fancy
\renewcommand{\headrulewidth}{0pt} % customise the layout...
\lhead{}\chead{}\rhead{}
\lfoot{}\cfoot{\thepage}\rfoot{}

%%% SECTION TITLE APPEARANCE
\usepackage{sectsty}
\allsectionsfont{\sffamily\mdseries\upshape} 
\usepackage[nottoc,notlof,notlot]{tocbibind} % Put the bibliography in the ToC
\usepackage[titles,subfigure]{tocloft} % Alter the style of the Table of Contents
\renewcommand{\cftsecfont}{\rmfamily\mdseries\upshape}
\renewcommand{\cftsecpagefont}{\rmfamily\mdseries\upshape}

\begin{document}

\tableofcontents
\newpage
\section{EWMA chart}
\begin{itemize}
\item In statistical quality control, the EWMA chart (or exponentially-weighted moving average chart) is a type of control chart used to monitor either variables or attributes-type data using the monitored business or industrial process's entire history of output.
\item While other control charts treat rational subgroups of samples individually, the EWMA chart tracks the exponentially-weighted moving average of all prior sample means. 
\item EWMA weights samples in geometrically decreasing order so that the most recent samples are weighted most highly while the most distant samples contribute very little.
\item 
Although the normal distribution is the basis of the EWMA chart, the chart is also relatively robust in the face of non-normally distributed quality characteristics.
\item There is, however, an adaptation of the chart that accounts for quality characteristics that are better modeled by the Poisson distribution.
\item The chart monitors only the process mean; monitoring the process variability requires the use of some other technique.
\item 
The EWMA control chart requires a knowledgeable person to 
select two parameters before setup:
\item 
The first parameter is $\lambda$, the weight given to the most recent rational subgroup mean. $\lambda$ must satisfy $0 < \lambda \leq 1$, but selecting the "right" value is a matter of personal preference and experience. 
%One source recommends 0.05 ≤ λ ≤ 0.25,[2]:411 while another recommends 0.2 ≤ λ ≤ 0.3.[1]
\item The second parameter is L, the multiple of the rational subgroup standard deviation that establishes the control limits. L is typically set at 3 to match other control charts, but it may be necessary to reduce L slightly for small values of $\lambda$.
\item Instead of plotting rational subgroup averages directly, the EWMA chart computes successive observations zi by computing the rational subgroup average, $\bar{x_i}$, and then combining that new subgroup average with the running average of all preceding observations, zi - 1, using the specially–chosen weight, $\lambda$, as follows:

\[z_i = \lambda \bar x_i + \left ( 1 - \lambda \right)z_{i - 1}\]
\item The control limits for this chart type are 

\[T \pm L\frac {S}{\sqrt n}\sqrt{\frac{\lambda}{2 - \lambda}\lbrack 1 - \left ( 1 - \lambda \right )^{2i} \rbrack}\] where T and S are the estimates of the long-term process mean and standard deviation established during control-chart setup and n is the number of samples in the rational subgroup. Note that the limits widen for each successive rational subgroup, approaching \[ \pm L\frac {\hat \sigma}{\sqrt n}.\]

\item The EWMA chart is sensitive to small shifts in the process mean, but does not match the ability of Shewhart-style charts (namely the $\bar{x}$ and R and $\bar{x}$ and s charts) to detect larger shifts.
\item One author recommends superimposing the EWMA chart on top of a suitable Shewhart-style chart with widened control limits in order to detect both small and large shifts in the process mean.
\end{itemize}
%-------------------------------------------------------------------------------------------%
\newpage
\section{Multivariate EWMA Control Chart}
\begin{itemize}
\item MEWMA is the natural multivariate extension of the EWMA chart proposed by
Roberts (1959). It was introduced by Lowry et al. (1992) and is more sensible in
detecting nonrandom changes in the process and based on the principle of the
weighted average of the previously observed vectors.
\item
Despite the fact that it is used mainly for individual observations (n = 1) it can
be utilized in rational subgroup case as it will be explained later. It is also a chart for
Phase II.
\end{itemize}
\end{document}