Lean Six Sigma is a methodology that relies on a collaborative team effort to improve performance by systematically removing waste;[1] combining lean manufacturing/lean enterprise and Six Sigma to eliminate the eight kinds of waste (muda): Time, Inventory, Motion, Waiting, Over production, Over processing, Defects, and Skills (abbreviated as 'TIMWOODS').

The Lean Six Sigma concepts were first published in a book titled Lean Six Sigma: Combining Six Sigma with Lean Speed by Michael George and Robert Lawrence Jr. in 2002. Lean Six Sigma utilises the DMAIC phases similar to that of Six Sigma. Lean Six Sigma projects comprise aspects of Lean's waste elimination and the Six Sigma focus on reducing defects, based on critical to quality (CTQ) characteristics.[clarification needed] The DMAIC toolkit of Lean Six Sigma comprises all the Lean and Six Sigma tools. The training for Lean Six Sigma is provided through the belt based training system similar to that of Six Sigma. The belt personnel are designated as white belts, yellow belts, green belts, black belts and master black belts, similar to karate.

For each of these belt levels skill sets are available that describe which of the overall Lean Six Sigma tools are expected to be part at a certain Belt level. These skill sets provide a detailed description of the learning elements that a participant will have acquired after completing a training program. The level upon which these learning elements may be applied is also described. The skill sets reflects elements from Six Sigma, Lean and other process improvement methods like the theory of constraints (TOC) total productive maintenance (TPM).
