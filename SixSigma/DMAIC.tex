Define[edit]
The purpose of this step is to clearly articulate the business problem, goal, potential resources, project scope and high-level project timeline. This information is typically captured within project charter document. Write down what you currently know. Seek to clarify facts, set objectives and form the project team. Define the following:

A problem
The customer(s)
Voice of the customer (VOC) and Critical to Quality (CTQs) — what are the critical process outputs?
The target process subject to DMAIC and other related business processes
Project targets or goal
Project boundaries or scope
A project charter is often created and agreed upon during the Define step.
Measure[edit]
The purpose of this step is to objectively establish current baselines as the basis for improvement. This is a data collection step, the purpose of which is to establish process performance baselines. The performance metric baseline(s) from the Measure phase will be compared to the performance metric at the conclusion of the project to determine objectively whether significant improvement has been made. The team decides on what should be measured and how to measure it. It is usual for teams to invest a lot of effort into assessing the suitability of the proposed measurement systems. Good data is at the heart of the DMAIC process:

Identify the gap between current and required performance.
Collect data to create a process performance capability baseline for the project metric, that is, the process Y(s) (there may be more than one output).
Assess the measurement system (for example, a gauge study) for adequate accuracy and precision.
Establish a high level process flow baseline. Additional detail can be filled in later.
Analyze[edit]
The purpose of this step is to identify, validate and select root cause for elimination. A large number of potential root causes (process inputs, X) of the project problem are identified via root cause analysis (for example a fishbone diagram). The top 3-4 potential root causes are selected using multi-voting or other consensus tool for further validation. A data collection plan is created and data are collected to establish the relative contribution of each root causes to the project metric, Y. This process is repeated until "valid" root causes can be identified. Within Six Sigma, often complex analysis tools are used. However, it is acceptable to use basic tools if these are appropriate. Of the "validated" root causes, all or some can be

List and prioritize potential causes of the problem
Prioritize the root causes (key process inputs) to pursue in the Improve step
Identify how the process inputs (Xs) affect the process outputs (Ys). Data is analyzed to understand the magnitude of contribution of each root cause, X, to the project metric, Y. Statistical tests using p-values accompanied by Histograms, Pareto charts, and line plots are often used to do this.
Detailed process maps can be created to help pin-point where in the process the root causes reside, and what might be contributing to the occurrence.
Improve[edit]
The purpose of this step is to identify, test and implement a solution to the problem; in part or in whole. This depends on the situation. Identify creative solutions to eliminate the key root causes in order to fix and prevent process problems. Use brainstorming or techniques like Six Thinking Hats and Random Word. Some projects can utilize complex analysis tools like DOE (Design of Experiments), but try to focus on obvious solutions if these are apparent. However, the purpose of this step can also be to find solutions without implementing them.

Create
Focus on the simplest and easiest solutions
Test solutions using Plan-Do-Check-Act (PDCA) cycle
Based on PDCA results, attempt to anticipate any avoidable risks associated with the "improvement" using FMEA
Create a detailed implementation plan
Deploy improvements
Control[edit]
The purpose of this step is to sustain the gains. Monitor the improvements to ensure continued and sustainable success. Create a control plan. Update documents, business process and training records as required.

A Control chart can be useful during the Control stage to assess the stability of the improvements over time by serving as 1. a guide to continue monitoring the process and 2. provide a response plan for each of the measures being monitored in case the process becomes unstable.
