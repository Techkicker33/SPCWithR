Six Sigma 
Six Sigma at many organizations simply means a measure of quality that strives for near perfection. Six Sigma is a disciplined,  data-driven approach and methodology for eliminating defects (driving 
toward six standard deviations between the mean and the nearest specification limit) in any process – from manufacturing to transactional and from product to service. The statistical representation of Six Sigma describes quantitatively how a process is performing. To achieve Six Sigma, a process must not produce more than 3.4 defects per million opportunities. A Six Sigma defect is defined as anything outside of customer specifications. A Six Sigma opportunity is then the total quantity of chances for a defect. The fundamental objective of the Six Sigma methodology is the implementation of a measurement-based strategy that focuses on process improvement and variation reduction through the application of Six Sigma improvement projects. 
This is accomplished through the use of two Six Sigma sub-methodologies: DMAIC and DMADV.  The Six Sigma DMAIC process (define, measure, analyze, improve, control) is an improvement system for existing processes falling below specification and looking for incremental improvement. 
The Six Sigma DMADV process (define, measure, analyze, design, verify) is an improvement system used to develop new processes or products at Six Sigma quality levels. It can also be employed if a current process requires more than just incremental improvement.
Both Six Sigma processes are executed by Six Sigma Green Belts and Six Sigma Black Belts, and are overseen by Six Sigma Master Black Belts.


Linear Models

•	Simple Linear Regression 
o	Checking Model Assumptions
o	Inferences
o	Deming Regression / Orthogonal Regression / Error in Variable Models
o	Prediction Intervals
	Equivalence of tests for slope and correlation 
o	Diagnostics
o	Residuals
•	Multiple Linear Regression
o	Revision of Linear Algebra
o	The Hat Matrix
o	Leave one out (Sherman-Woodbury)
•	Variable Selection Procedures
o	Law of Parsimony
o	Stepwise Regression
o	Forward Selection
