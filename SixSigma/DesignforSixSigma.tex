Design for Six Sigma (DFSS) is a business-process management "methodology" related to traditional Six Sigma.[1] It is used in many industries, like finance, marketing, basic engineering, process industries, waste management, and electronics. It is based on the use of statistical tools like linear regression and enables empirical research similar to that performed in other fields, such as social science. While the tools and order used in Six Sigma require a process to be in place and functioning, DFSS has the objective of determining the needs of customers and the business, and driving those needs into the product solution so created. DFSS is relevant for relatively simple items / systems. It is used for product or process design in contrast with process improvement.[2][3] Measurement is the most important part of most Six Sigma or DFSS tools, but whereas in Six Sigma measurements are made from an existing process, DFSS focuses on gaining a deep insight into customer needs and using these to inform every design decision and trade-off.

There are different options for the implementation of DFSS. Unlike Six Sigma, which is commonly driven via DMAIC (Define - Measure - Analyze - Improve - Control) projects, DFSS has spawned a number of stepwise processes, all in the style of the DMAIC procedure.[4]

DMADV, define – measure – analyze – design – verify, is sometimes synonymously referred to as DFSS, although alternatives such as IDOV (Identify, Design, Optimize, Verify) are also used. The traditional DMAIC Six Sigma process, as it is usually practiced, which is focused on evolutionary and continuous improvement manufacturing or service process development, usually occurs after initial system or product design and development have been largely completed. DMAIC Six Sigma as practiced is usually consumed with solving existing manufacturing or service process problems and removal of the defects and variation associated with defects. It is clear that manufacturing variations may impact product reliability. So, a clear link should exist between reliability engineering and Six Sigma (quality). In contrast, DFSS (or DMADV and IDOV) strives to generate a new process where none existed, or where an existing process is deemed to be inadequate and in need of replacement. DFSS aims to create a process with the end in mind of optimally building the efficiencies of Six Sigma methodology into the process before implementation; traditional Six Sigma seeks for continuous improvement after a process already exists.
