

\documentclass[SPC-MASTER.tex]{subfiles}
\begin{document}

%\tableofcontents
\Large
%----------------------------------------------------%
\section{Deming's Lectures (source unknown)}

This article attempts to clarify the role played by W. Edwards Deming at the beginning of the modern Japanese quality control movement by summarizing and analyzing the actual content of the series of quality control lectures he gave in Japan during the summer of 1950. The primary source documents are the published lecture transcripts that Deming considered authentic.Analysis of the transcripts shows that Deming spent most of the eight-day lecture series discussing statistical process control. 


However, he opened the lectures with extended remarks that contain a core of the philosophy for which he later became famous. Yet, significant elements of what is now known as the Deming method or Deming philosophy did not appear in the lecture series. Deming included in the lectures an extended discussion of sampling inspection that revealed his ambivalence to the subject. The transcripts show that Deming introduced to the Japanese a product design cycle of Shewhart that is distinct from the management process that the Japanese later came to call the plan-do-check-act cycle.

\end{document}