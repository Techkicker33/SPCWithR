\documentclass{beamer}

\usepackage{framed}
\usepackage{graphicx}

\begin{document}
	
%	
%\subsection{Control Charts}
% http://jjmcd.fedorapeople.org/Download/R/R-2.13-Six_Sigma_with_R_-_A_Tutorial-en-US.pdf
%------------------------------------------------ %
\begin{frame}
\frametitle{Control Charts}
\begin{itemize}
\item During the Measure phase, one of the first things the Back Belt wants to do is to determine whether the
process is in control with respect to the major 'Y'. 
\item The primary tool for this is a control chart. 
\end{itemize}
%\item In many
%cases, the process may already keep control charts; many do. But there are large number of way in which
%control charts are produced, and a great many pitfalls, so the Black Belt would be well advised to examine
%the procedures used for the control chart and ensure they are appropriate for his purposes.
 
\end{frame}
%------------------------------------------------ %
\begin{frame}
	\frametitle{Control Charts}
	\begin{itemize}
		\item
The simplest control chart consists of a simple plot of the observed variable versus time, with the control
limits marked on the chart, and sometimes, the specification limit.

\item The control limits are typically set at +/- three standard deviations. It is important to remember that the
control limits should not be recalculated each time the control chart is redrawn. 
\item Rather, they should be set
once, and then changed because of a change in the process.
\end{itemize} 
\end{frame}
%------------------------------------------------ %
%------------------------------------------------ %
\begin{frame}
\frametitle{Control Charts}
\begin{itemize}
\item The control chart is a graph used to study how a process changes over time. 
\item Data are plotted in time order. A control chart always has a central line for the average, an upper line for the upper control limit and a lower line for the lower control limit.
\item These lines are determined from historical data.
\end{itemize} 
\end{frame}
%------------------------------------------------ %
\begin{frame}
\frametitle{Control Charts}
By comparing current data to these lines, you can draw conclusions about whether the process variation is consistent (in control) or is unpredictable (out of control, affected by special causes of variation).
\end{frame}
%------------------------------------------------ %
\begin{frame}
\frametitle{Control Charts}
\begin{itemize}
\item Control charts for variable data are used in pairs. The top chart monitors the average, or the centering of the distribution of data from the process. 
\item The bottom chart monitors the range, or the width of the distribution. If your data were shots in target practice, the average is where the shots are clustering, and the range is how tightly they are clustered. 
\item Control charts for attribute data are used singly.
\end{itemize}

\end{frame}
%------------------------------------------------ %
\begin{frame}
\frametitle{When to Use a Control Chart}

\begin{itemize} 
\item When controlling ongoing processes by finding and correcting problems as they occur.
\item When predicting the expected range of outcomes from a process.
\item When determining whether a process is stable (in statistical control).
\item When analyzing patterns of process variation from special causes (non-routine events) or common causes (built into the process).
\item When determining whether your quality improvement project should aim to prevent specific problems or to make fundamental changes to the process.
\end{itemize}
\end{frame}
%------------------------------------------------ %

\end{document}
