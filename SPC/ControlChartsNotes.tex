Statistical Process Control (SPC)  
Statistical Process Control (SPC) is an industry-standard methodology for measuring and controlling quality during the manufacturing process. Quality data in the form of Product or Process measurements are obtained in real-time during manufacturing. 
This data is then plotted on a graph with pre-determined control limits. Control limits are determined by the capability of the process, whereasspecification limits are determined by the client's needs.

Control Limits on an XBar Range Chart
Data that falls within the control limits indicates that everything is operating as expected. Any variation within the control limits is likely due to a common cause—the natural variation that is expected as part of the process. 
If data falls outside of the control limits, this indicates that an assignable cause is likely the source of the product variation, and something within the process should be changed to fix the issue before defects occur.


\documentclass{beamer}

\usepackage{framed}
\usepackage{graphicx}

\begin{document}
%------------------------------------------------ %
\begin{frame}
\frametitle{Control Charts}
\begin{itemize}
\item The control chart is a graph used to study how a process changes over time. 
\item Data are plotted in time order. A control chart always has a central line for the average, an upper line for the upper control limit and a lower line for the lower control limit.
\item These lines are determined from historical data.
\end{itemize} 
\end{frame}
%------------------------------------------------ %
\begin{frame}
\frametitle{Control Charts}
By comparing current data to these lines, you can draw conclusions about whether the process variation is consistent (in control) or is unpredictable (out of control, affected by special causes of variation).
\end{frame}
%------------------------------------------------ %
\begin{frame}
\frametitle{Control Charts}
\begin{itemize}
\item Control charts for variable data are used in pairs. The top chart monitors the average, or the centering of the distribution of data from the process. 
\item The bottom chart monitors the range, or the width of the distribution. If your data were shots in target practice, the average is where the shots are clustering, and the range is how tightly they are clustered. 
\item Control charts for attribute data are used singly.
\end{itemize}

\end{frame}
%------------------------------------------------ %
\begin{frame}
\frametitle{When to Use a Control Chart}

\begin{itemize} 
\item When controlling ongoing processes by finding and correcting problems as they occur.
\item When predicting the expected range of outcomes from a process.
\item When determining whether a process is stable (in statistical control).
\item When analyzing patterns of process variation from special causes (non-routine events) or common causes (built into the process).
\item When determining whether your quality improvement project should aim to prevent specific problems or to make fundamental changes to the process.
\end{itemize}
\end{frame}
%------------------------------------------------ %

\end{document}
