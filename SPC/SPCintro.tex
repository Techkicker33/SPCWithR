%% - http://www.personal.soton.ac.uk/jav/soton/HELM/workbooks/workbook_46/46_2_quality_control.pdf
%% - http://www3.ul.ie/~mlc/support/BusStats/Notes/
\documentclass[12pt]{article}
\usepackage{framed}
\usepackage{amsmath}
\usepackage{amssymb}
\usepackage{graphics}
\begin{document}
\tableofcontents

%---------------------------------- %
\newpage
\section{Statistical Process Control}

\begin{itemize}
\item Commonly used in the manufacturing process, statistical process control (SPC) makes use of statistical facts gleaned through statistical analysis to both monitor and control virtually any process where output can be measured. SPC makes use of a variety of tools inherent to the method to include experimentation, control charts and continuous improvement processes. 


\item The key difference between SPC and other process control methods is a focus on quantitative analysis, rather than opinion, when analyzing variations in a process. Applied to a wide range of processes aside from manufacturing, statistical process control focuses on identifying sources of variation and determining the extent of that variation. Based on such information, 
managers can make decisions regarding whether the variation is acceptable, if it signifies a problem or a positive causation that needs replicating.


\item Beginning with the premise that any output that is measurable will have variation from either common, natural causes or special, assignable causes, statistical process control seeks to determine if a variation is under statistical control. Using control charts, analysts will look for variations in a process during the time period that chart specifies. 

\item Upon identifying those variations, the analyst will then use the chart to determine the origin of the variation and whether that variation is within a pre-determined, specified range. 
\begin{itemize}
\item When identified variations fall within a predetermined, specified range, the process is defined as being \textbf{under statistical control}. 
\item If not, however, the process is then considered as being \textbf{out of statistical control}.
\end{itemize}

\item 
Variations that are found to be out of statistical control are said to originate from special, assignable causes. Such variations are usually determined by the actual process, and statistical software is often used to perform the required calculations, which are subsequently plotted on the \textbf{control chart}. 

\item
Statistical process control aims to determine if a process in under statistical control, because if it is then the process and be predicted. Accurately predicting the outputs of a process provides analysts with important information, such as how long it will take to fulfill a specific type of production order. Thereafter, the concern with the SPC method is getting the process back under statistical control so that outputs can be predicted reliably.

\item
Once a process is determined to be out of statistical control, assignable causes are searched for and determined whether they are positive or negative to the process.

\item 

Negative causes are addressed after investigation to ascertain and eliminate the causation, and then the process is reanalyzed repeatedly with SPC until the problem is fixed. Positive causes usually follow the same process, but with the objective of implementing the causation at all times in the process.
\end{itemize}
%---------------------------------- %
\newpage
\section{Process Control Theory}
\begin{itemize}
\item Process control theory establishes methods of observing and correcting variances between desired and actual output. Statistics play a major role in process control, since statistical methods are used to determine acceptable limits and deviations from an ideal average. Engineering processes establish certain quality standards to increase efficiency, create a safe work environment, and ensure product consistency.

\item Manufacturing environments often set up automated control processes based on process control theory. The foundation of the theory states that quality can be improved by reducing performance inconsistencies through mathematical control methods. A manufacturing facility's management works with a company's executives to determine ideal product attributes, which are used in inspection checkpoints and as measurements of quality. \\
\textbf{Important} - One of the main goals behind process control is to reduce extreme variations within the same finished product.

\item By seeking to reduce deviations from an established norm, applications of process control theory help increase cost efficiency. Machinery used in the manufacturing of goods can be programmed to produce certain end results and product characteristics, which saves companies time and money. Even though it is not practical to completely automate some production processes, workers can also use statistical process control methods. 
\item This is typically seen with random batch inspections of finished products, as quality control teams have to make decisions on whether to modify automation, scrap an entire batch, or allow completed products to go to market.

\item At times variations in product or performance consistency can be attributed to \textbf{uncontrollable circumstances}. These factors are usually uncovered when sharp deviations occur between desired and actual performance. Since indications of these variances often show up in reported statistics, further investigation typically occurs. Part of process control theory is determining the root cause of inconsistencies and finding probable ways to correct them.

\item Process control theory recognizes that some circumstances that lead to undesired results are extreme and uncontrollable. In the case of a manufacturing facility, a natural disaster or power outage might qualify as an \textbf{uncontrollable circumstance} behind interrupted production. 
\item Practical applications of the theory help managers identify why deviations may be occurring since many causes can be attributed to \textbf{controllable factors}, such as inadequate materials, outdated machinery, incorrect parameters, and poor training methods. 
\item  One of the important aspects behind process control is establishing acceptable high and low limits.

These limits define an acceptable range of deviation from the \textbf{ideal average}. 
\item Most business leaders recognize and accept the fact that there will always be some degree of variance between desired and actual performance. The idea is to reduce the variance as much as possible, most often keeping the quality of all products within \textbf{two (or three) standard deviations} from the \textbf{established norm}.
\end{itemize}


\end{document}

