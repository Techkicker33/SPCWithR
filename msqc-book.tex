

\documentclass[11pt]{article} % use larger type; default would be 10pt

\usepackage[utf8]{inputenc}

%%% PAGE DIMENSIONS
\usepackage{geometry} % to change the page dimensions
\geometry{a4paper} % or letterpaper (US) or a5paper or....
% \geometry{margin=2in} % for example, change the margins to 2 inches all round
% \geometry{landscape} % set up the page for landscape
%   read geometry.pdf for detailed page layout information

\usepackage{graphicx} % support the \includegraphics command and options

% \usepackage[parfill]{parskip} % Activate to begin paragraphs with an empty line rather than an indent

%%% PACKAGES
\usepackage{booktabs} % for much better looking tables
\usepackage{array} % for better arrays (eg matrices) in maths
\usepackage{paralist} % very flexible & customisable lists (eg. enumerate/itemize, etc.)
\usepackage{verbatim} % adds environment for commenting out blocks of text & for better verbatim
\usepackage{subfig} % make it possible to include more than one captioned figure/table in a single float
% These packages are all incorporated in the memoir class to one degree or another...

%%% HEADERS & FOOTERS
\usepackage{fancyhdr} % This should be set AFTER setting up the page geometry
\pagestyle{fancy} % options: empty , plain , fancy
\renewcommand{\headrulewidth}{0pt} % customise the layout...
\lhead{}\chead{}\rhead{}
\lfoot{}\cfoot{\thepage}\rfoot{}

%%% SECTION TITLE APPEARANCE
\usepackage{sectsty}
\allsectionsfont{\sffamily\mdseries\upshape} 
\usepackage[nottoc,notlof,notlot]{tocbibind} % Put the bibliography in the ToC
\usepackage[titles,subfigure]{tocloft} % Alter the style of the Table of Contents
\renewcommand{\cftsecfont}{\rmfamily\mdseries\upshape}
\renewcommand{\cftsecpagefont}{\rmfamily\mdseries\upshape}

\begin{document}

\tableofcontents
\newpage

\section{Motivation of Talk}

% Introduction fro Book

\begin{itemize}
\item The intensive use of an automatic data acquisition systems and the use of
on-line computers for process monitoring have led to an increased occurrence of
industrial processes with two or more correlated quality characteristics, in which
the statistical process control and the capability analysis should be performed using
multivariate methodologies. (book)
\item Unfortunately, despite the availability of increased computing capabilities, in
the Multivariate Statistical Quality Control (MSQC) framework the software
solutions are limited or restricted in their level of success and ease of use for
dealing with the problems of industry or promoting academic instruction.
\end{itemize}

\newpage
\section{Multivariate Control Charts}

\begin{itemize}
\item With the enhancements in data acquisition systems it is usual to deal with processes
with more than one correlated quality characteristic to be monitored. 
\item A common
practice is to control the stability of the process using univariate control charts.
\item This
practice increases the probability of false alarm of special cause of variation.
\end{itemize}
\newpage

\begin{itemize}
\item The control ellipsoid or $\chi^2$ control chart
\item The $T^2$ or Hotelling chart
\item The Multivariate Exponentially Weighted Moving Average (MEWMA) chart
\item The Multivariate Cumulative Sum (MCUSUM) chart
\item The chart based on Principal Components Analysis (PCA)
\end{itemize}


\newpage

\section{Hotelling $T^2$ Control Chart (Phase I)}
\begin{itemize}
\item The origin of the $T^2$ control chart dates back to the pioneer works of Harold Hotelling
who applied this method to the bombsight problem in Second World War.
\item  The
Hotelling (1947) procedure has become without doubt the most applied in multivariate
process control and it is the multivariate analogous of the Shewhart control chart.
\item For that reason, it is also known as multivariate Shewhart control chart.
\item Often in practice the parameters $\mu$ and $\Sigma$ are unknown and consequently must be estimated across the unbiased estimators $\bar{x}$ and S.
\end{itemize}

\newpage

\newpage
\section{Graphical Methods}
\begin{itemize}
\item The first section of this chapter will examine two graphical techniques: histogram
and Q-Q plot that facilitate the assumption of normality.
\item 
Histogram is a graphical technique that allows a visual summary of the data. It
provides information about the center, the spread, the skewness, and the existence
of outliers. (NIST / SEMATECH e-Handbook of Statistical Methods).
\item
A visual inspection of a histogram permits to establish an initial hypothesis of
the distribution; in this case a bell-shaped is desired.
\item
Although histograms are basically used in univariate scenarios, univariate
normality per se does not imply multivariate normality; if a departure from normality
is founded in individual variables, this has a negative effect in the
multinormality.
\end{itemize}
\end{document}
