%--------------------------------------------------------------- %
\documentclass[SPC-MASTER.tex]{subfiles}
\begin{document}
\Large
	
\section{Assumption of Normality}
\subsection{Multivariate Normal}
{\Large
\begin{itemize}
\item The multivariate normal distribution (or multivariate Gaussian distribution), is a generalization of the one-dimensional (univariate) normal distribution to higher dimensions.\item One possible definition is that a random vector is said to be k-variate normally distributed if every linear combination of its k components has a univariate normal distribution. 
%\item However, its importance derives mainly from the multivariate central limit theorem.
\item The multivariate normal distribution is often used to describe, at least approximately, any set of (possibly) correlated real-valued random variables each of which clusters around a mean value.
\end{itemize}
}
\begin{figure}[h!]
\centering
\includegraphics[width=0.8\linewidth]{./793px-MultivariateNormal}
\caption{}
\label{fig:793px-MultivariateNormal}
\end{figure}
%------------------------------------------------------ % 
\newpage
\subsection{Testing for Normality}
%MSQC
{
\large
\textbf{Graphical Methods}
\begin{itemize}
\item Histograms
\item Normal Probability Plots
\end{itemize}

\noindent\textbf{Hypothesis Tests for Univariate Data}
\begin{itemize}
\item Shapiro-Wilk Test (inbuilt with \texttt{R})
\item D'Agostino Test (MSQC package)
\end{itemize}
\bigskip

\noindent\textbf{Hypothesis Tests for Multivariate Data}
\begin{itemize}
\item Mardia Test (MSQC package)
\item Henze and Zirkler (MSQC package)
\item Royston Test (MSQC package)
\end{itemize}


}
\newpage
\subsubsection{The bimetal data set (MSQC package)}
{\large
\begin{itemize}
\item Bimetal thermostat has innumerable practical uses. These types of thermostats hold
a bimetallic strip composed by two strips of different metals that convert the
changing of temperature in mechanical displacement due to the difference in
thermal expansion.
\item Certain type of strip composed of brass and steel is analyzed in a quality
laboratory by testing the deflection, curvature, resistivity, and hardness in low
and high expansion sides.
\end{itemize}
{
\large

\begin{verbatim}
> tail(bimetal1)
      deflection curvature resistivity Hardness low side Hardness high side
[23,]      20.76     39.98       14.98             22.29             26.03
[24,]      21.00     40.11       15.17             22.04             25.99
[25,]      20.57     39.73       14.35             22.02             25.80
[26,]      20.78     39.83       15.27             21.60             25.89
[27,]      20.96     40.03       15.26             21.98             25.94
[28,]      21.14     39.93       14.98             21.84             25.98

\end{verbatim}
}
\newpage
\begin{figure}[h!]
\centering
\includegraphics[width=0.9\linewidth]{images/MSQC-bimetal1hist}
\caption{}
\label{fig:MSQC-bimetal1hist}
\end{figure}
\newpage
\begin{figure}[h!]
\centering
\includegraphics[width=0.9\linewidth]{images/MSQC-bimetal1qq}
\caption{}
\label{fig:MSQC-bimetal1qq}
\end{figure}
\newpage
\subsubsection{D'Agostino Test (MSQC Package)}
\begin{itemize}
\item Using the bimetal1 data set in MSQC package
\end{itemize}
\begin{framed}
\begin{verbatim}
> for (i in 1 : 5){
+  DAGOSTINO(bimetal1[,i])
+  }
D'Agostino Test
    Skewness
      Skewness coefficient: 0.0831225 
      Statistics: 0.2117358 
      p-value: 0.8323131 
    Kurtosis
      The kurtosis coefficient: 3.0422 
      Statistics: 0.591983 
      p-value: 0.553862 
    Omnibus Test
      Chi-squared: 0.3952759 
      Degree of freedom: 2
      p-value: 0.8206669 
....
....
D'Agostino Test
    Skewness
      Skewness coefficient: -0.04173762 
      Statistics: -0.1063873 
      p-value: 0.9152751 
    Kurtosis
      The kurtosis coefficient: 4.162062 
      Statistics: 1.675258 
      p-value: 0.09388364 
    Omnibus Test
      Chi-squared: 2.817807 
      Degree of freedom: 2
      p-value: 0.2444111 
\end{verbatim}
\end{framed}
\newpage
\subsubsection{Some Multivariate (MSQC Pacakge)}
\begin{framed}
\begin{verbatim}
> MardiaTest(bimetal1)
$skewness
[1] 6.982112

$p.value
[1] 0.585327

$kurtosis
[1] 33.77373

$p.value
[1] 0.3490892

> 
>
>
> HZ.test(bimetal1)
[1] 0.6068650 0.7709586
> 
> 
> Royston.test(bimetal1)
test.statistic        p.value 
     1.1814742      0.9364221 
\end{verbatim}
\end{framed}

\newpage


\subsubsection{Box Cox Transformation}
\begin{itemize}
\item The Box-Cox transforms nonnormally distributed data to a set of data that has approximately normal distribution. 
\end{itemize}
\end{document}